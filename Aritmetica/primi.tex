\subsection{Numeri primi e coprimi}
\begin{quote}
    \emph{Dicesi \textbf{primo} un numero divisibile solo per se stesso e per l'unità; i numeri non primi si dicono ``composti''.}
\end{quote}
Sappiamo che la serie dei numeri primi è illimitata: ad oggi, tuttavia, non è stata ancora trovata una funzione in grado di rappresentare l'intera serie dei numeri primi. Mersenne provò a definire i numeri primi con una funzione del tipo $n(p) = 2^{p}-1$, dove $p$, però, rappresenta un numero primo.\\\\
Quello di numero primo è uno dei concetti basilari della teoria dei numeri, la parte della matematica che studia i numeri interi: l'importanza sta nella possibilità di costruire con essi, attraverso la moltiplicazione, tutti gli altri numeri interi, nonché nell'unicità di tale fattorizzazione.\\
Essi sono inoltre oggetto di studio fin dall'antichità: i primi risultati risalgono agli antichi Greci e in particolare agli \emph{Elementi} di Euclide, scritti attorno al 300 a.C. Ciononostante, numerose congetture che li riguardano non sono state ancora dimostrate; tra le più note vi sono l'ipotesi di Riemann, la congettura di Goldbach e quella dei primi gemelli, ancora indimostrate nonostante i numerosi secoli trascorsi dalla loro formulazione.\\\\
Due numeri interi $a$ e $b$ sono invece detti \textbf{coprimi} (o primi tra loro) se e solo se essi non hanno nessun divisore comune eccetto 1 e -1 o, in modo equivalente, se il loro massimo comune divisore è 1.
\begin{equation}\label{coprimi}
\text{ MCD}(a,b)=1 \iff (a,b) \in \mathbb{Z}_{coprimi}
\end{equation}
La loro rilevanza è notevole in moltissimi ambiti della matematica pura, come l'algebra o la geometria; recentemente hanno assunto un'importanza cruciale anche nella matematica applicata, e in particolare nella crittografia.\\
Un metodo molto efficiente per determinare se due numeri sono coprimi è fornito dall'\textbf{algoritmo di Euclide}.