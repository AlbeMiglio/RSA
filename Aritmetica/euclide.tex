\subsection{Algoritmo di Euclide}
L'algoritmo di Euclide ha la funzione di trovare l'MCD tra due numeri interi. È uno degli algoritmi più antichi conosciuti, in quanto presente negli \emph{Elementi} euclidei; certamente era conosciuto da Eudosso di Cnido intorno al 375 a.C. e persino Aristotele ne ha fatto cenno ne \emph{I topici}.\\
L'algoritmo è particolarmente efficiente in quanto non richiede la fattorizzazione dei due numeri, la cui lentezza computazionale cresce con un andamento esponenziale.\\
Euclide originariamente formulò il problema geometricamente, per trovare una "misura" comune per la lunghezza di due segmenti, e il suo algoritmo procedeva sottraendo ripetutamente il più corto dal più lungo.\\\\
Partendo da due numeri naturali $a$ e $b$, è detto $r_i$ il resto della divisione (di passo $i$-esimo) $a_i \div b_i$ tale che $a_i=q_1\cdot b_i+r_i$. \\
Secondo il teorema alla base dell'algoritmo:
\begin{equation}
    \text{MCD}(a_i,b_i)= \text{MCD}(b_i,r_i)
\end{equation}
iterando il calcolo del resto per ciascun passo, si otterrà con un numero $N$ finito di passi l'ultimo resto $r_N$ non nullo che coinciderà con l'MCD$(a,b)$.\\
Sappiamo che il numero di passi da compiere è certamente finito poichè il resto di una divisione è sempre strettamente minore del divisore, e pertanto i numeri
$r_i$,$r_{i+1}$,$r_{i+2}$,... formano una sequenza strettamente decrescente di numeri interi positivi, che deve
prima o poi raggiungere il valore 0.\\\\
Scrivendo i due numeri $a$ e $b$ come multipli del loro MCD (che per semplicità chiameremo $d$) e la loro divisione:\\\\
$\left \{ \begin{array}{rl}
a=d \cdot k_a
\\b=d \cdot k_b
\end{array}
\right. \Rightarrow r=d \cdot k_a - q (d \cdot k_b) \Rightarrow r = d (k_a - q \cdot k_b);$\\\\
appare evidente come il resto sia anch'esso multiplo di MCD$(a,b)$. Andando a ritroso nell'algoritmo si nota come per ciascun resto $r_n$ valga la relazione $(r_n|r_{n-1})$. Di fatto ciascun resto $r_n > 0$ è dunque multiplo di MCD$(a,b)$ e poichè l'ultimo resto è il minore di tutti, il minore dei multipli di MCD$(a,b)$ sarà proprio l'MCD stesso.
\begin{equation}
    a,b \in \mathbb{Z} \,\implies \, \exists \, N\in \mathbb{Z} \mid r_N = \text{MCD}(a,b)
\end{equation}
Tenendo nota dei quozienti ottenuti durante lo svolgimento dell'algoritmo, si possono determinare due interi $p$ e $q$ tali che $ap+bq=\text{MCD}(a,b)$. Questo però è noto con il nome di algoritmo di Euclide esteso, la cui applicazione più interessante consiste nel calcolo dei coefficienti dell'identità di B\'ezout.\\