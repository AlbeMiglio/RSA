\subsection{Decriptazione}
Una volta che il destinatario riceve il messaggio cifrato $c$, questi è in grado di decifrarlo grazie alla propria chiave privata $(n,d)$.\\
Ricordando l'identità di B\'ezout \eqref{Bezout} possiamo scrivere:
\begin{equation}
\begin{split}
1 \equiv ed+k\varphi(n) \text{ (mod $n$)} & \Longrightarrow\\
& \Longrightarrow m \equiv m^{ed} \cdot m^{k\varphi(n)} \text{ (mod $n$)}
\end{split}
\end{equation}
e, ricordando il teorema di Euler-Fermat \eqref{EulerFermat}, semplifichiamo:
\begin{equation}
\begin{split}
m \equiv m^{ed} \cdot m^{k\varphi(n)} \text{ (mod $n$)} & \Longrightarrow\\
& \Longrightarrow m \equiv \left(m^{e}\right)^{d} \cdot \left(m^{\varphi(n)}\right)^{k} \text{ (mod $n$)} \Longrightarrow \\
& \Longrightarrow m \equiv \left(m^{e}\right)^{d} \cdot \left(1\right)^{k} \text{ (mod $n$)} \Longrightarrow \\
& \Longrightarrow m \equiv \left(c\right)^{d} \text{ (mod $n$)};
\end{split}
\end{equation}
Elevando a potenza con esponente $d$ in modulo $n$ il messaggio cifrato $c$, quindi, otterremo il messaggio di partenza $m$.\\
L'operazione è possibile in quanto $n$, come già ribadito, è pubblico mentre $d$ è noto al destinatario (e solo a lui).