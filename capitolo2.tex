\section{Struttura dell'algoritmo}
L'algoritmo RSA è definito a chiave pubblica: significa che ciascun membro di uno scambio di dati possiede una chiave pubblica, ossia nota a tutti, finalizzata alla cifratura del dato da comunicare, e una chiave privata, segreta, necessaria per la decifratura del messaggio stesso.\\
Le chiavi, come vedremo a breve, sono generate in maniera tale che un messaggio cifrato con una chiave pubblica possa essere decifrato solo con la sua chiave privata \textbf{e viceversa}.\\
Poichè le chiavi pubbliche sono note a tutti, i messaggi decifrabili con esse garantiscono l'autenticità del mittente, in quanto sicuramente cifrate con le loro univoche chiavi private; viceversa, poichè le chiavi private sono segrete, i messaggi cifrati con la chiave pubblica del destinatario possono essere decifrati solo con la sua chiave privata, pertanto esse garantiscono la segretezza del messaggio, che potrà essere decifrato solo dal destinatario.\\\\
L'algoritmo ha 4 fasi principali:
\begin{enumerate}
    \item Generazione delle chiavi: è il passaggio più ``intenso'', in cui vengono effettivamente svolti i calcoli maggiori;
    \item Distribuzione della chiave pubblica: è un passaggio necessario per distribuire ai membri della comunicazione la chiave di cifratura del messaggio;
    \item Criptazione: il messaggio da diffondere, grazie alle chiavi generate in precedenza, viene ``criptato'' (dal greco \textkappa \textrho \textupsilon \textpi \texttau \textomikron \textvarsigma, ``nascosto'');
    \item Decriptazione: il messaggio ricevuto viene interpretato e trasformato, per invertire la cifratura e riportare il contenuto voluto dal mittente;
\end{enumerate}
\subsection{Generazione delle chiavi}
La generazione delle chiavi prevede cinque fasi principali:
\begin{enumerate}
    \item Si scelgono arbitrariamente due numeri $p$ e $q$, che dovranno rimanere segreti e pertanto non dovranno mai essere comunicati;
    \item Se ne calcola il prodotto $n=pq$, che sarà di pubblico dominio;
    \item Si applica la funzione $\varphi$ a $n$: il valore di $\varphi(n)$ dovrà rimanere segreto;
    \item Si sceglie arbitrariamente un numero $e$ (pubblico) coprimo con $\varphi(n)$ e tale che $1 < e < \varphi(n)$;
    \item Si determina l'inverso $d$ (privato) di $e$ \text{(mod $\varphi(n)$)}, con i metodi descritti in precedenza nella \eqref{coefficienti}; poichè $e$ ed $n$ soddisfano le condizioni della \eqref{CoprimiInvertibili}, abbiamo la certezza dell'esistenza dell'inverso in modulo $n$.
\end{enumerate}
A questo punto si dispone di tutte le informazioni necessarie per la distribuzione delle chiavi.\\
La chiave pubblica sarà infatti costituita dalla coppia:
\begin{equation*}
(n,e)
\end{equation*}
mentre la chiave privata dalla coppia:
\begin{equation*}
(n,d)
\end{equation*}
\subsection{Distribuzione della chiave}
La distribuzione è una fase fondamentale per l'interazione con gli altri membri dello scambio di dati. \'E ad ogni modo utile ricordare che non è necessario distribuire la chiave pubblica con metodi sicuri, proprio in quanto pubblica per definizione. Va ribadita tuttavia l'importanza dell'utilizzo di un metodo che garantisca che il messaggio inviato giunga a destinazione.
\subsection{Criptazione}
Una volta convertito, con criteri arbitrariamente definiti, il testo del messaggio da trasmettere in un numero $m$ tale che $0 \leq m < n$, si può cifrare con la chiave pubblica del \textbf{destinatario} il messaggio con una semplice elevamento a potenza in modulo:
\begin{equation}\label{cifratura}
    c \equiv m^e \text{ (mod $n$)}
\end{equation}
Si osserva che l'operazione è possibile in quanto il messaggio $m$ da inviare è noto al mittente per definizione e la coppia di numeri $(n,e)$ è pubblica.
\subsection{Decriptazione}
Una volta che il destinatario riceve il messaggio cifrato $c$, questi è in grado di decifrarlo grazie alla propria chiave privata $(n,d)$.\\
Ricordando l'identità di B\'ezout \eqref{Bezout} possiamo scrivere:
\begin{equation}
\begin{split}
1 \equiv ed+k\varphi(n) \text{ (mod $n$)} & \Longrightarrow\\
& \Longrightarrow m \equiv m^{ed} \cdot m^{k\varphi(n)} \text{ (mod $n$)}
\end{split}
\end{equation}
e, ricordando il teorema di Euler-Fermat \eqref{EulerFermat}, semplifichiamo:
\begin{equation}
\begin{split}
m \equiv m^{ed} \cdot m^{k\varphi(n)} \text{ (mod $n$)} & \Longrightarrow\\
& \Longrightarrow m \equiv \left(m^{e}\right)^{d} \cdot \left(m^{\varphi(n)}\right)^{k} \text{ (mod $n$)} \Longrightarrow \\
& \Longrightarrow m \equiv \left(m^{e}\right)^{d} \cdot \left(1\right)^{k} \text{ (mod $n$)} \Longrightarrow \\
& \Longrightarrow m \equiv \left(c\right)^{d} \text{ (mod $n$)};
\end{split}
\end{equation}
Elevando a potenza con esponente $d$ in modulo $n$ il messaggio cifrato $c$, quindi, otterremo il messaggio di partenza $m$.\\
L'operazione è possibile in quanto $n$, come già ribadito, è pubblico mentre $d$ è noto al destinatario (e solo a lui).